

\usepackage{graphicx}
\usepackage{subfigure}
% \usepackage{overpic}
\usepackage{grffile} % for image names with spaces or dots

% color support
\usepackage{color}

% for animation
\usepackage{multimedia}
\usepackage{animate}



% definitions
% \useoutertheme{shadow}
\setbeamertemplate{blocks}[rounded][shadow=true] %blocks round and shadowed
\setbeamertemplate{navigation symbols}{} %navigation symbols off
% \usecolortheme[named=gray]{structure} %farbe des Text in boxen


\usepackage{thumbpdf}
\usepackage{wasysym}
\usepackage{ucs}
\usepackage[utf8]{inputenc}
\usepackage{verbatim}
\usepackage[percent]{overpic}

\usepackage{multimedia}
\usepackage{animate}

\usepackage{tikz}
\usepackage{pgf,pgfarrows,pgfnodes,pgfautomata,pgfheaps,pgfshade}
\usetikzlibrary{arrows,shapes}

% ######### inserted for overlay function
% \usepackage{tikz}
% \usepackage{multimedia}
% \usetikzlibrary{mindmap,trees}
% \usetikzlibrary{shapes} 
% \usetikzlibrary{snakes}
% \usetikzlibrary{calc}
% ###########

\usetikzlibrary{calendar} % LTEX and plain TEX


\tikzstyle{every picture}+=[remember picture]

\usepackage[absolute,overlay]{textpos} 
\setlength{\TPHorizModule}{1cm}
\setlength{\TPVertModule}{\TPHorizModule}
\textblockorigin{1cm}{1cm} % start everything near the top-left corner
\setlength{\parindent}{0pt}

% um text etc. zu drehen
\usepackage{rotating}

% % % in case you want to change the colors separately:
\definecolor{greenMW}{RGB}{0,60,0} %another shade of green to use
\setbeamercolor{frametitle}{fg=greenMW} %color of the title in \begin{frame}{frame title}

% \setbeamercolor{title}{parent=palette tertiary}
\setbeamercolor{block title}{bg=greenMW, fg=lightgray} %background (bg) and foreground (fg) color of the block title
\setbeamercolor{block body}{bg=lightgray!60} %bg color of block body

% alert block settings
\setbeamercolor{block title alerted}{bg=green!80!black, fg=gray} %bg/fg color of title
\setbeamercolor{block body alerted}{bg=green!35!black, fg=lightgray} %bg/fg color of block body

% example block settings
\setbeamercolor{block title example}{bg=greenMW!70!black, fg=lightgray} %bg/fg color of title
\setbeamercolor{block body example}{bg=darkred!70!black, fg=lightgray} %bg/fg color of block body


% below the coloring of the parts which are used for section/subsection etc. in the header or footer of a slide can be entered
\setbeamercolor*{section in head/foot}{fg=lightgray,bg=greenMW}
\setbeamercolor*{subsection in head/foot}{fg=blue!35!black,bg=lightgray!95!blue}
\setbeamercolor*{title in head/foot}{fg=blue!35!black,bg=lightgray!95!blue}
\setbeamercolor*{title}{fg=lightgray,bg=greenMW}
\setbeamercolor*{author in head/foot}{fg=lightgray,bg=greenMW}
\setbeamercolor*{date in head/foot}{fg=lightgray,bg=blue!35!black}


\setbeamercolor{item}{fg=greenMW}





% % ########
% % ## tikz helper 
% % #########
% 
% 
% %% helper macros
% \newcommand\pgfmathsinandcos[3]{%
%   \pgfmathsetmacro#1{sin(#3)}%
%   \pgfmathsetmacro#2{cos(#3)}%
% }
% \newcommand\LongitudePlane[3][current plane]{%
%   \pgfmathsinandcos\sinEl\cosEl{#2} % elevation
%   \pgfmathsinandcos\sint\cost{#3} % azimuth
%   \tikzset{#1/.estyle={cm={\cost,\sint*\sinEl,0,\cosEl,(0,0)}}}
% }
% \newcommand\LatitudePlane[3][current plane]{%
%   \pgfmathsinandcos\sinEl\cosEl{#2} % elevation
%   \pgfmathsinandcos\sint\cost{#3} % latitude
%   \pgfmathsetmacro\yshift{\cosEl*\sint}
%   \tikzset{#1/.estyle={cm={\cost,0,0,\cost*\sinEl,(0,\yshift)}}} %
% }
% \newcommand\DrawLongitudeCircle[2][1]{
%   \LongitudePlane{\angEl}{#2}
%   \tikzset{current plane/.prefix style={scale=#1}}
%    % angle of "visibility"
%   \pgfmathsetmacro\angVis{atan(sin(#2)*cos(\angEl)/sin(\angEl))} %
%   \draw[current plane,thin,black] (\angVis:1) arc (\angVis:\angVis+180:1);
%   \draw[current plane,thin,dashed] (\angVis-180:1) arc (\angVis-180:\angVis:1);
% }%this is fake: for drawing the grid
% \newcommand\DrawLongitudeCirclered[2][1]{
%   \LongitudePlane{\angEl}{#2}
%   \tikzset{current plane/.prefix style={scale=#1}}
%    % angle of "visibility"
%   \pgfmathsetmacro\angVis{atan(sin(#2)*cos(\angEl)/sin(\angEl))} %
%   \draw[current plane,green!30!black,thick] (150:1) arc (150:180:1);
%   %\draw[current plane,dashed] (-50:1) arc (-50:-35:1);
% }%for drawing the grid
% \newcommand\DLongredd[2][1]{
%   \LongitudePlane{\angEl}{#2}
%   \tikzset{current plane/.prefix style={scale=#1}}
%    % angle of "visibility"
%   \pgfmathsetmacro\angVis{atan(sin(#2)*cos(\angEl)/sin(\angEl))} %
%   \draw[current plane,black,dashed, ultra thick] (150:1) arc (150:180:1);
% }
% \newcommand\DLatred[2][1]{
%   \LatitudePlane{\angEl}{#2}
%   \tikzset{current plane/.prefix style={scale=#1}}
%   \pgfmathsetmacro\sinVis{sin(#2)/cos(#2)*sin(\angEl)/cos(\angEl)}
%   % angle of "visibility"
%   \pgfmathsetmacro\angVis{asin(min(1,max(\sinVis,-1)))}
%   \draw[current plane,dashed,black,ultra thick] (-50:1) arc (-50:-35:1);
% 
% }
% \newcommand\fillred[2][1]{
%   \LongitudePlane{\angEl}{#2}
%   \tikzset{current plane/.prefix style={scale=#1}}
%    % angle of "visibility"
%   \pgfmathsetmacro\angVis{atan(sin(#2)*cos(\angEl)/sin(\angEl))} %
%   \draw[current plane,green!30!black,thin] (\angVis:1) arc (\angVis:\angVis+180:1);
% 
% }
% 
% \newcommand\DrawLatitudeCircle[2][1]{
%   \LatitudePlane{\angEl}{#2}
%   \tikzset{current plane/.prefix style={scale=#1}}
%   \pgfmathsetmacro\sinVis{sin(#2)/cos(#2)*sin(\angEl)/cos(\angEl)}
%   % angle of "visibility"
%   \pgfmathsetmacro\angVis{asin(min(1,max(\sinVis,-1)))}
%   \draw[current plane,thin,black] (\angVis:1) arc (\angVis:-\angVis-180:1);
%   \draw[current plane,thin,dashed] (180-\angVis:1) arc (180-\angVis:\angVis:1);
% }%Defining functions to draw limited latitude circles (for the red mesh)
% \newcommand\DrawLatitudeCirclered[2][1]{
%   \LatitudePlane{\angEl}{#2}
%   \tikzset{current plane/.prefix style={scale=#1}}
%   \pgfmathsetmacro\sinVis{sin(#2)/cos(#2)*sin(\angEl)/cos(\angEl)}
%   % angle of "visibility"
%   \pgfmathsetmacro\angVis{asin(min(1,max(\sinVis,-1)))}
%   %\draw[current plane,red,thick] (-\angVis-50:1) arc (-\angVis-50:-\angVis-20:1);
% \draw[current plane,green!30!black,thick] (-50:1) arc (-50:-35:1);
% 
% }
% 
% \tikzset{%
%   >=latex,
%   inner sep=0pt,%
%   outer sep=2pt,%
%   mark coordinate/.style={inner sep=0pt,outer sep=0pt,minimum size=3pt,
%     fill=black,circle}%
% }

%% document-wide tikz options and styles

% \tikzset{%
%   >=latex, % option for nice arrows
%   inner sep=0pt,%
%   outer sep=2pt,%
%   mark coordinate/.style={inner sep=0pt,outer sep=0pt,minimum size=3pt,
%     fill=black,circle}%
% }


% #######


% \setbeamercovered{transparent}
% \setbeamercovered{dynamic}
% \setbeamercovered{invisble}

% to display source code on the slides:
\usepackage{listings}
% \lstset{language=R}
\lstset{basicstyle=\footnotesize,numbers=left, numberstyle=\tiny, numbersep=2pt,language=R,commentstyle=\tiny\textcolor{gray!80!blue},showstringspaces=false,breaklines=true,breakatwhitespace=false,backgroundcolor=\color{lightgray!50!blue}}
% more infos: http://en.wikibooks.org/wiki/LaTeX/Packages/Listings
\lstset{ %
language=bash,                % choose the language of the code
basicstyle=\scriptsize,       % the size of the fonts that are used for the code
numbers=left,                   % where to put the line-numbers
numberstyle=\tiny,      % the size of the fonts that are used for the line-numbers
stepnumber=1	,                   % the step between two line-numbers. If it's 1 each line will be numbered
numbersep=5pt,                  % how far the line-numbers are from the code
backgroundcolor=\color{gray!20},  % choose the background color. You must add \usepackage{color}
showspaces=false,               % show spaces adding particular underscores
showstringspaces=false,         % underline spaces within strings
showtabs=false,                 % show tabs within strings adding particular underscores
frame=single,	                % adds a frame around the code
% framerule=1pt,
tabsize=2,	                % sets default tabsize to 2 spaces
captionpos=b,                   % sets the caption-position to bottom
breaklines=true,                % sets automatic line breaking
breakatwhitespace=true,        % sets if automatic breaks should only happen at whitespace
escapeinside={\%*}{*)}          % if you want to add a comment within your code
}



% path to graphics
\usepackage{graphics}
\graphicspath{{pics/}{logo/}}

